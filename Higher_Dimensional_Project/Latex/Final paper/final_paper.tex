\documentclass[12pt]{article}
\usepackage[latin1]{inputenc}
\textheight 240mm
\textwidth  170mm
\oddsidemargin  0mm
\evensidemargin 0mm
\topmargin -20mm
\usepackage{amsmath}
\usepackage{amsfonts}
\usepackage{amssymb}
\usepackage{graphicx}
\usepackage{cite}
\graphicspath{{"C:/Users/Johnny_Five/Google Drive/ECEN5322/Final_project_data/Report_pictures/"}} 

\author{Jeff Venicx}
\title{Music classification methods}
\begin{document}
\maketitle
\newpage

\section{Introduction}
\section{Pre processing}
Since the raw data has been given in a WAV format which is not conducive to analyzing it must first be broken down into more manageable pieces of data that can then be used for more advanced algorithms.
In order to be able too effectively work with the huge amount of data given we must be able to reduce the dimensionality. There are a variety of ways in which to do this which will be explained in the next section.
\subsection{MFCCs}
MFCCs try to represent the audio spectrum similar to a human ear's perception making it one of the most powerful features for classifying music. MFCCs can be generated by the following steps. First step is to divide the audio signal into frames of typically 20ms. Next a windowing function is applied to get rid of the edge effects. Then DFT is performed on the frames and take the logarithm f the amplitude spectrum. The resultant spectrum is then converted to a Mel Spectrum.  The Mel Scale is actually a mapping between actual frequency and frequency perceived by the human ear. A human ear doesn't perceive audio in a linear manner. It perceives it as linear below 1 Khz and logarithmic above. The components of the mel spectral vectors are then de-correlated using a DCT. %~\cite{Pampalk}
\subsection{zero crossing}
It is the average rate at which the song crosses the zero amplitude and it is an excellent feature to distinguish jazz from hard pop. The higher the ZCR, greater the noisiness of the track.
\section{Dimensionality Reduction}
 With the preprocessing techniques described the data has been turned into a more useful and dimensionally smaller form. In order to further reduce the dimensionality and ability to classify music the follow techniques were used. 
\subsection{Random Projection}
 Using Johnson Lindenstrauss we estimated to what dimension the tracks can be brought down to and then we performed uniform sampling.(go more in detail here about our technique)

\subsection{Fluctuation Patterns}
Fluctuation patterns (FPs) allow for the analysis of the loudness per frequency band. This has the advantage of describing characteristics of a piece of audio that cannot be described with the spectral similarity of a track. The basic steps used to calculate this data include cutting the data into smaller segments, applying an FFT to find the amplitude modulations frequencies in within the lower audio bands, some filtering to emphasize certain patterns. This allows for the songs to be processed and show the fluctuations in the intensity of a song. %~\cite{Pampalk}

\subsection{Gaussian Mixture Model}
The use of Gaussian Mixture Models(GMMs) consists of clustering the frames and finding the Gaussian distributions of them. This is done with an expectation maximization algorithm in conjunction with A multivariate Gaussian probability density function. This technique allows for fine tuning of the amount dimensions left because we can pick any number of Gaussian for our data to be represented with. For our purposes we used 30,5 and a single Gaussian distribution. 

\subsection{LLE}
LLE , constructs a graph representation of the data points. In contrast with other techniques such as ISOMAP or multidimensional scaling, it preserves the local geometric properties.
\subsection{ISOmap}
ISOMAP, is a technique that addresses issues caused by classical scaling. Classical scaling mainly aims to maintain pairwise Euclidean distance when the datapoints could actually be lying on a curved manifold. ISOMAP addresses this issue by attempting to preserve geodesic distance(curvilinear) between data points.

\section{Distance Calculations}
To make a prediction of which genre a song belongs to a method of measuring the distances between song is necessary. We use a few different methods to accomplish this. 
\subsection{Euclidean Norm}
This is the simplest of techniques used it takes the dimensionally reduced data and measure the differences in the sum squared.
\[\sqrt{\sum_{i=1}^{n} (q_{i}-p_{i})^{2}} \]
\subsection{K nearest neighbors}
\includegraphics[width = \textwidth]{accuracy_vs_k-nearest.jpg}
\subsection{Monte Carlo}
\subsection{K-L divergence}

\section{Neural Nets}

\section{Results and Methods}
\subsection{random projection and Knn}
\subsection{MFCC, 30GMM, Monte , NN5}
	
\includegraphics[width = \textwidth]{cms_mfcc20_similarity.jpg}

\newpage

\begin{center}
	Average = 74.6\\
	Deviation = 5.5\\
	\hfill \\
	Average Values by Genre\\
\begin{tabular}{|c|c|c|c|c|c|}
	\hline
	Classical & Electronic& Jazz/Blues& Metal/Punk&Rock/Pop&World\\
	\hline
	\textbf{87.9}&	0.9&	0&	0&	4.3&	22.8 \\
	\hline
	0.9&	\textbf{69.8}&	0&	0.3&	13.4&	4 \\
	\hline 	
	0.8&	2.3&	\textbf{97.3}&	0&	5.4&	0.3 \\
	\hline
	0.2&	1.2&	0&	\textbf{74.7}&	8.6&	0 \\
	\hline
	0.5&	9&	2.7&	21.9&	\textbf{46.8}&	1.1 \\	
	\hline
	9.7&	16.8&	0&	3.1&	21.5&	\textbf{71.8} \\		
	\hline
\end{tabular}

\hfill \break

	Standard Deviation by Genre\\
	\begin{tabular}{|c|c|c|c|c|c|}
		\hline
		Classical & Electronic& Jazz/Blues& Metal/Punk&Rock/Pop&World\\
		\hline
		\textbf{3.8}&	1.7&	0&	0&	3.7&	16.9 \\
		\hline
		0.9&	\textbf{8.7}&	0&	1.8&	5.3&	6.6 \\		
		\hline
		 1.2&	3&	\textbf{14.8}&	0&	3.6&	1.8 \\			
		\hline
		0.5&	2&	0&	\textbf{16.4}&	5.3&	0 \\		
		\hline
		0.8&	5.9&	14.8&	14&	\textbf{9.3}&	3.2 \\	
		\hline
		3.5&	7.1&	0&	6.1&	6.7&	\textbf{17.1} \\				
		\hline
	\end{tabular}
	
\end{center}

\subsection{FP, Euclidian, NN5}


\subsection{MFCC20 ISOmap(5NN, 20D) Euclidean, NN5}

\includegraphics[width = \textwidth]{"ISOmap distance best"}
\newpage
\begin{center}
	Average = 48.2\\
	Deviation = 6.7\\
	\hfill \\
	Average Values by Genre\\
	\begin{tabular}{|c|c|c|c|c|c|}
		\hline
		Classical & Electronic& Jazz/Blues& Metal/Punk&Rock/Pop&World\\
		\hline
		\textbf{74.4}&	8.3&	31.5&	5.3&	8.6&	37.2 \\	
		\hline
		3.6&	\textbf{44.6}&	5.2&	11&	21.2&	20.3 \\
		\hline 	
		2.7&	2.3&	\textbf{62.5}&	2.5&	5&	2.2 \\
		\hline
		0.8&	4.4&	0&	\textbf{38.6}&	12.3&	1.6 \\
		\hline
		3.4&	20&  	0&	36.6&		\textbf{37.1}&	6.1 \\
		\hline
		15.1&	20.4&	0.8&	6&	15.8&	\textbf{32.6} \\	
		\hline
	\end{tabular}
	
	\hfill \break
	
	Standard Deviation by Genre\\
	\begin{tabular}{|c|c|c|c|c|c|}
		\hline
		Classical & Electronic& Jazz/Blues& Metal/Punk&Rock/Pop&World\\
		\hline	
		\textbf{4.7}&	3.8&	32.2&	7.2&	5.5&	11.1 \\
		\hline
		2.4&		\textbf{8.5}&	18&	9&	8.1&0	10.5\\
		\hline 	
		1.8&	3.1&	\textbf{34.8}&	4.5&	4&	4.2 \\
		\hline
		0.9&	3.5&	0&	\textbf{12.9}&	7.3&0	3.4 \\
		\hline
		2.1&	8.6&	0&	13.2&	\textbf{10.7}&	5.9 \\
		\hline
		4&	7.6&	5.1&	7.2&	7.6&	\textbf{10.3} \\
		\hline
	\end{tabular}
\end{center}


\subsection{MFCC20-FP-Euclidian, NN5}
\begin{center}
	Average = 58.3\\
	Deviation = 6.8\\
	\hfill \\
Average Values by Genre\\
\begin{tabular}{|c|c|c|c|c|c|}
	\hline
	Classical & Electronic& Jazz/Blues& Metal/Punk&Rock/Pop&World\\
	\hline
	\textbf{76.6}&	0&	8.4&	0.9&	0.8&	19.2 \\
	\hline
	1.6&	\textbf{91.4}&	12.2&	14.2&	18.8&	15.3 \\
	\hline 	
	2.9&	0&	\textbf{50}&	0&	3.3&	6 \\
	\hline
	0.3&	0&	7.7&	\textbf{44.8}&	16.4&	2.6 \\
	\hline
	3.9&	3.6&	0.7&	38.5&	\textbf{42.7}&	13.3 \\
	\hline
	14.7&	5&	21&	1.7&	18&	\textbf{43.7}\\
	\hline
\end{tabular}

\hfill \break

Standard Deviation by Genre\\
\begin{tabular}{|c|c|c|c|c|c|}
	\hline
	Classical & Electronic& Jazz/Blues& Metal/Punk&Rock/Pop&World\\
	\hline	
	\textbf{4.5}&	0&	19.6&	3.2&	1.9&	10 \\
	\hline
	1.2&	\textbf{9.1}&	17.5&	13.5&	7.2&	9.1 \\
	\hline 	
	1.8&	0&	\textbf{37.2}&	0&	3.3&	6.2 \\
	\hline
	0.5&	0&	15.4&	\textbf{18.1}&	6.7&	3.9 \\
	\hline
	1.7&	4.8&	4.9&	16.1&	\textbf{8.6}&	8.2 \\
	\hline
	3.5&	7&	29.2&	4.4&	6.2&	\textbf{12.1} \\
	\hline
\end{tabular}
\end{center}






\subsection{mfcc, GMM1, neuralnet 1 layer 20 neurons}


\begin{center}
	Average total = 67.7\\
	Average Deviation = 6.7\\
	\hfill \\
	Average Values by Genre\\
\begin{tabular}{|c|c|c|c|c|c|}
	\hline
	Classical & Electronic& Jazz/Blues& Metal/Punk&Rock/Pop&World\\
	\hline
		\textbf{86.9}&	2.1&	8.6&	2.1&	4.8&	17.9 \\	
	\hline
		0.7&	\textbf{65.6}&	1.7&	  3&	14.2&	 15 \\
	\hline 	
		0.3&	0.4&	\textbf{80.4}&	0.5&	1.5&	3.6 \\
	\hline
		0.7&	1&		0.5&	\textbf{65.5}&	10&		2 \\
	\hline
		1.9&	13.2&	1.8&	21.9&	\textbf{56.9}&	10.4\\	
	\hline
		9.4&	17.7&	7.1&	7&	12.6&	\textbf{51.1} \\		
	\hline
\end{tabular}
 
 \hfill \break


	Standard Deviation by Genre\\
	\begin{tabular}{|c|c|c|c|c|c|}
		\hline
		Classical & Electronic& Jazz/Blues& Metal/Punk&Rock/Pop&World\\
		
		
		
		\hline	
		\textbf{4.6}&	3.7&	13.3&	6.2&	5.8&	9.5 \\
		\hline		
		1.2&    \textbf{11.1}&	8.3&	7.4&	9.3&	8.1\\
		\hline 			
		0.9&	1.4&	\textbf{19.1}&	2.4&	3.3&	4.9 \\
		\hline		
		1&	  	2.7&	3.5&	\textbf{21.1}&	8.5&	3.6 \\
		\hline	
		1.7&	8.4&	6.1&	18&		\textbf{11.9}&	7.3 \\
		\hline		
		3.8&	8.9&	13.6&	8.9&	8&	\textbf{12.9} \\		
		\hline
	\end{tabular}
\end{center}


\section{Algorithm Design}



\section{Classification of new data}

\section{Conclusion}

~\cite{Nobody06}



\bibliography{mybib}{}
\bibliographystyle{plain}


\end{document}